% Generated by GrindEQ Word-to-LaTeX 
\documentclass{book} % use \documentstyle for old LaTeX compilers

\usepackage[utf8]{inputenc} % 'cp1252'-Western, 'cp1251'-Cyrillic, etc.
\usepackage[english]{babel} % 'french', 'german', 'spanish', 'danish', etc.
\usepackage{amsmath}
\usepackage{amssymb}
\usepackage{txfonts}
\usepackage{mathdots}
\usepackage[classicReIm]{kpfonts}
\usepackage{graphicx}

% You can include more LaTeX packages here 


\begin{document}

%\selectlanguage{english} % remove comment delimiter ('%') and select language if required


\noindent \textbf{Proyecto Final de seguridad y Auditoria}

\noindent \textbf{Universidad Mariano G\'{a}lvez Sede de Boca del Monte}

\noindent \textbf{David Alejandro Serrano Salazar 7690-13-19355}

\noindent \textbf{Ervin Gabriel Laferre Guevara 7690-16-10153}

\noindent \textbf{Guatemala, 06 de noviembre de 2021}

\noindent \textbf{}

\noindent \textbf{}

\noindent \textbf{Configuraci\'{o}n de las maquinas asignadas:}

\noindent 

\noindent Cada una de las maquinas virtuales que instalamos est\'{a}n relacionadas con una IP est\'{a}tica que se le dio por medio del PFSENSE que actuara tanto como nuestro servidor de VPN como nuestro FIREWALL.

\noindent 

\noindent \includegraphics*[width=1.72in, height=1.14in]{image1}

\noindent \textbf{Servidor FREEDSB}

\noindent Como vemos tiene una IP asignada por medio de nuestra VPN como su IP local.

\noindent 

\noindent \includegraphics*[width=4.50in, height=2.97in, trim=0.00in 0.05in 0.07in 0.07in]{image2}

\noindent 

\noindent \textbf{CONFIGURACION DEL FIREWALL.}

\noindent Como vemos existen varios usuarios que se crearon para tener distintos accesos y niveles de seguridad dependiendo el tipo de usuario que tengamos, asimismo se entablaron grupos que son administrador para que se puedan modificar desde cualquier maquina y no solo se dependa de una persona para realizar dicha configuraci\'{o}n, en esta pantalla para agregar permisos bastar\'{i}a con agregar (bot\'{o}n verde) y la ventana que levanta ingresar los datos requeridos y seleccionar si es administrador o no.

\noindent 

\noindent \includegraphics*[width=6.10in, height=2.90in]{image3}

\noindent 

\noindent \textbf{CONFIGURACION DE REGLAS.}

\noindent Para configurar las reglas dentro del firewall nos vamos a la barra de men\'{u} y encontraremos la parte de FIREWALL, pinchamos en ella. Y nos desplegara una tabla con todas las reglas configuradas a cada uno de los usuarios, donde hemos podido denegar el trafico especifico a una p\'{a}gina, as\'{i} como a una secci\'{o}n de IPS para prohibir el uso de redes sociales, la cual esta regla se puede activar a una cantidad infinita de usuario, tambi\'{e}n hemos permitido el acceso y denegar el acceso a la pagina de la UMG con la finalidad de probar esta.

\noindent \includegraphics*[width=6.37in, height=3.14in]{image4}

\noindent 

\noindent \textbf{CONFIGURACION DE IP PUBLICA}

\noindent La configuraci\'{o}n de cada una de estas IPS es necesaria para el funcionamiento de la VPN, con la cual, una nos servir\'{a} para la red interna que crear\'{a} la VPN y la otra nos servir\'{a} para la salida del internet.

\noindent 

\noindent \includegraphics*[width=6.12in, height=3.08in]{image5}

\noindent \textbf{}

\noindent \textbf{}

\noindent \textbf{CONFIGURACION DE IP PRIVADA}

\noindent Ya una vez dentro creamos reglas de comportamiento. Una sola regla activa para todos lados de ida y vuelta la configuraci\'{o}n, como puede ser el caso del internet, o puertos espec\'{i}ficos para conexiones, tal es el caso de algunos servicios como Microsoft Teams, Meet, Skype entre otros.  

\noindent 

\noindent \includegraphics*[width=6.00in, height=2.80in, trim=0.09in 0.09in 0.00in 0.00in]{image6}

\noindent 

\noindent \textbf{CONFIGURACION DE IP PRIVADA}

\noindent Cada vez que realizamos una actualizaci\'{o}n del firewall sin importar la acci\'{o}n que estemos realizando nos encontraremos con el siguiente escenario al momento de darle guardar para activar las reglas nos saldr\'{a} un mensaje de confirmaci\'{o}n para validar que estemos de acuerdo y si lo estamos nos saldr\'{a} el mensaje que vemos a continuaci\'{o}n, indicando que la reglas han sido aplicadas inmediatamente se puede visualizar en los clientes estos cambios.

\noindent \includegraphics*[width=5.63in, height=2.33in, trim=0.36in 0.34in 0.09in 0.00in]{image7}

\noindent \textbf{PRUEBAS DE LAS REGLAS IMPLEMENTADAS }

\noindent Como describimos en la parte de arriba el servidor tiene la IP con terminaci\'{o}n 7 en las reglas del FIREWALL, indicamos que se pod\'{i}a enviar paquetes ICMP, como podemos visualizar en la siguiente imagen

\noindent 

\noindent \includegraphics*[width=4.94in, height=2.28in]{image8}

\noindent Luego de esto realizamos una actualizaci\'{o}n de esta regla activando una restricci\'{o}n completa a la red que no puedan enviar este tipo de paquetes, como vemos en la imagen continua no ha recibido ning\'{u}n paquete, todos se perdieron.

\noindent \includegraphics*[width=5.29in, height=1.93in]{image9}

\noindent 

\noindent \textbf{CREACION DE REGLAS DENTRO DEL FIREWALL}

\noindent En el apartado de Firewall, existe un bot\'{o}n color verde que indica agregar (ADD), en el cual nos dar\'{a} una ventana donde podemos configurar las reglas que creamos necesarias e indicar el usuario o los usuarios que afectar\'{a} esta regla, en la imagen siguiente, veremos el ejemplo de la creaci\'{o}n de una de estas para que la cualquier IP pueda visualizar el SERVIDOR ADD.

\noindent \includegraphics*[width=5.92in, height=2.73in, trim=0.09in 0.32in 0.08in 0.00in]{image10}

\noindent \includegraphics*[width=6.08in, height=2.28in]{image11}

\noindent \includegraphics*[width=6.08in, height=1.18in]{image12}

\noindent Una vez llenamos todos los campos que hemos visto en la imagen, pinchamos en el bot\'{o}n de guardar, y nos regresara a la pantalla anterior donde estar\'{a}n todas las reglas listadas dentro de la tabla, est\'{a} actualmente creada aparecer\'{a} desactivada y lo \'{u}nico que tendr\'{i}amos que realizar es la activaci\'{o}n de dicha regla, para que este vigente. 

\noindent \includegraphics*[width=6.01in, height=2.54in, trim=0.00in 0.13in 0.15in 0.00in]{image13}

\noindent 

\noindent Como vimos anteriormente el mensaje de aplicar cambios nos aparecer\'{a} y luego se activar\'{a}, otro ejemplo que pudimos realizar es la restricci\'{o}n de internet a la IP que se conecte en este caso bloqueamos la IP de Google.

\noindent  \includegraphics*[width=6.14in, height=1.48in]{image14}

\noindent 

\noindent Luego la desactivamos y vemos que es sencillo realizar reglas en el firewall para limitar el acceso a internet o alg\'{u}n segmento especifico de la red.

\noindent 

\noindent \includegraphics*[width=6.11in, height=2.08in]{image15}

\noindent 


\end{document}

